% -*- coding=utf-8 -*-
%%%%%%%%%%%%%%%%%%%%%%%%%%%%%%%%%%%%%%%%%%%%%%%%%%%%%%%%%%%%%%%%%%%%%%%%%%%%%%%%%%%%%%%%%%%%%%%
%   陕师大硕士学位论文模板
%   由overleaf上的陕师大学位论文模板修改制作而成
%   使用要求:
%       编译器:Texlive2019或以上
%       编辑器:Texstudio
%   使用方法:
%       编辑/tex目录下的各个tex文件的文件内容
%       最后编译main.tex
%       警告:需要修改配置就编辑main.tex的内容,编辑内容去/tex目录下的各个文件
%   输出 docx查重: 
% 		参考链接: https://zhuanlan.zhihu.com/p/455713759
% 		软件:pandoc
% 		在项目主目录下运行如下命令
%             g7714-2005
% 			pandoc main.tex  --filter pandoc-crossref --citeproc --csl csl/chinese-gb7714-2005-numeric --bibliography=bib/ref.bib -M reference-section-title=Reference  -M autoEqnLabels -M tableEqns  -t docx+native_numbering --number-sections -o output.docx
%             g7714-2015
% 			pandoc main.tex  --filter pandoc-crossref --citeproc --csl csl/china-national-standard-gb-t-7714-2015-numeric --bibliography=bib/ref.bib -M reference-section-title=Reference  -M autoEqnLabels -M tableEqns  -t docx+native_numbering --number-sections -o output.docx
% 		注意: Word 无法显示矢量图,提前将文中使用的矢量图转为 .PNG 格式。
% 		tips:查重不用放图,因此,可以将所有的图都画成矢量的
%   创建日期:2022/3/31
%   模板维护:Chalie
%   更新:根据2022年3月新出的论文格式进行修改
% 	内容结构:
% 		文档类型,宏包管理,页面边距,页眉页脚,章节标题,目录设置,参考文献,定理环境,
% 		图表环境,代码环境,引用工具,其余设定,正文内容
%%%%%%%%%%%%%%%%%%%%%%%%%%%%%%%%%%%%%%%%%%%%%%%%%%%%%%%%%%%%%%%%%%%%%%%%%%%%%%%%%%%%%%%%%%%%%%%

%=================================文档类型=====================================================
%	毕业论文选取ctexbook比较合适
%	twoside命令,设置为双面排版,左右页边距会根据奇偶页自动调整
%	12pt,字体大小,默认为10pt
%	openright命令,默认openright,即为新的一章在右手边开始

\documentclass[zihao=-4,a4paper,twoside,fontset=none, openright, UTF8, AutoFakeBold]{ctexbook}	% 文档类型

%=================================宏包管理=====================================================
%	和配置有关的宏包在具体的配置区引用,这里只引用正文区用到的宏包

\usepackage{wallpaper}	% 封面背景包
\usepackage{amsmath,mathtools,amsthm,amsfonts,amssymb,bm}	% AMS包
\usepackage{xeCJK,xeCJKfntef}

%%============================字体设置========================%%%
\setmainfont{Times New Roman}        %缺省英文字体 Times New Roman
\setCJKmainfont[Path=configuration/]{simsun.ttf}          %宋体
\setCJKsansfont[Path=configuration/]{simhei.ttf}          %黑体
\setCJKmonofont[Path=configuration/]{simsun.ttf}          %宋体
% ------------------------------------------------------------------------
\setCJKfamilyfont{song}[Path=configuration/]{simsun.ttf}  %导入宋体字体
\newcommand{\songti}{\CJKfamily{song}} %设置宋体快捷命令
%------------------------------------------------------------------------
\setCJKfamilyfont{kai}[Path=configuration/]{simkai.ttf}   %导入楷书字体
\newcommand{\kaishu}{\CJKfamily{kai}}   %设置楷书快捷命令
%------------------------------------------------------------------------
\setCJKfamilyfont{hei}[Path=configuration/]{simhei.ttf}   %导入黑体字体
\newcommand{\heiti}{\CJKfamily{hei}}   %设置黑体快捷命令

%=================================页面边距=====================================================
%	geometry宏包使用教程:http://www.ctex.org/documents/packages/layout/geometry.htm
%	A4纸宽210mm,长297mm
%	left + right + textwidth = 210
%	top + bottom + textheight = 297
%	headheight:页眉文字高度,应当小于等于top
\usepackage{geometry}		% 页面边距包
\geometry{%
	a4paper,
	left=30mm,
	right=30mm,
	top=25mm,
	bottom=25mm,
	textheight=247mm,
	textwidth=150mm,
	headheight=21.7mm
}

%=================================页眉页脚=====================================================
%	fancy宏包使用教程:http://www.ctex.org/documents/packages/layout/fancyhdr.htm
%	fancypagestyle{样式名}可以自定义样式,并通过\pagestyle{样式名}和\thispagestyle{样式名}来使用
%   \leftmark可以获取不带星号的chapter标题内容,\rightmark可以获取到不带星号的section标题内容
%   L, C, R分别表示左中右,
%   E, O分别表示偶数页和奇数页
\usepackage{fancyhdr}								% 页眉页脚包
\fancypagestyle{myfancy}{%
	\fancyhf{}	                                    % 清空所有定义
	\fancyfoot[CE,CO]{\thepage}                     % 设置页脚为当前页码
	\fancyhead[CE]{陕西师范大学硕士学位论文}          % 设置偶数页居中的页眉为陕西师范大学硕士学位论文
	\fancyhead[CO]{\leftmark}                       % 设置奇数页居中的页眉为当前章节名
	\renewcommand{\headrule}{%                      % 重定义headrule来实现双页眉装饰线效果
	    \makebox[0pt][l]{\rule[.7\baselineskip]{\headwidth}{3pt}}%
	    \rule[.6\baselineskip]{\headwidth}{0.4pt}\vskip-.8\baselineskip
    }
}

%=================================章节标题=====================================================
\ctexset{
	chapter = {%
		name = {第, 章},
		number =\arabic{chapter},  % 用阿拉伯数字显示章节号
		format += {\heiti  \zihao{-2} \centering},     % chapter格式添加一条:居中
		beforeskip = 10pt,			% 设置章节标题前的垂直间距为10pt,默认为50pt
		afterskip = 20pt,			% 设置章节标题后的垂直间距为20pt,默认为40pt
		fixskip = true,				% 设置固定间距为true,抑制标题前后的多余间距
	},
	section = {%
		format += {\zihao{3}\raggedright } ,   % section格式添加一条:左对齐
        beforeskip = 1em,			% 设置章节标题前的垂直间距为10pt,默认为50pt
        afterskip = 1em,			% 设置章节标题后的垂直间距为20pt,默认为40pt
        fixskip = true,				% 设置固定间距为true,抑制标题前后的多余间距
	},
	subsection = {%
		format += {\zihao{-3} \raggedright },    % subsection格式添加一条:左对齐
        beforeskip = 1em,			% 设置章节标题前的垂直间距为10pt,默认为50pt
        afterskip = 1em,			% 设置章节标题后的垂直间距为20pt,默认为40pt
        fixskip = true,				% 设置固定间距为true,抑制标题前后的多余间距
	},
	subsection = {%
        format += {\zihao{-4} \raggedright } ,   % subsection格式添加一条:左对齐
        beforeskip = 1em,			% 设置章节标题前的垂直间距为10pt,默认为50pt
        afterskip = 1em,			% 设置章节标题后的垂直间距为20pt,默认为40pt
        fixskip = true,				% 设置固定间距为true,抑制标题前后的多余间距
    }
}

%=================================目录设置=====================================================
%	titletoc宏包使用教程:https://blog.csdn.net/golden1314521/article/details/39926135
%						 https://blog.csdn.net/l_changyun/article/details/87431805
%
\usepackage{titletoc}		% 目录定制包
\titlecontents%	章
	{chapter}[4em]
	{\heiti\vspace*{7pt}}
	{\contentslabel{4em}}
	{\hspace*{-4em}}
	{~\titlerule*[0.6pc]{$.$}~\contentspage}
\titlecontents%	节
	{section}[4em]
	{}
	{\contentslabel{2em}}
	{\hspace*{-2em}}
	{~\titlerule*[0.6pc]{$.$}~\contentspage}
\titlecontents%	小节
	{subsection}[7em]
	{}
	{\contentslabel{3em}}
	{\hspace*{-2em}}
	{~\titlerule*[0.6pc]{$.$}~\contentspage}

%=================================参考文献=====================================================
%   biblatex宏包使用教程:
%	https://www.overleaf.com/learn/latex/Bibliography_management_with_biblatex

\usepackage[%
	backend=biber,				% 设置使用biber进行编译,也可以使用bibtex,但是功能更少
	style=gb7714-2015,			% 设置风格样式为国家标准gb7714-2015
	%sorting=ynt,					% 设置排序按照年份,名字,标题进行排序
	gbnamefmt=lowercase,
	gbpub=false,
	gbpunctin=false,
	]{biblatex}       			% 参考文献包
\addbibresource{bib/ref.bib}    % 加载参考文献的文件

%=================================定理环境=====================================================
% 	自定义定理类环境(定义,引理,定理,推论,例,注)
% 		定理环境命令:\newtheorem{name}[counter]{text}[section]
% 			name:		标识这个环境的关键字(用于编程)
%			counter:	(可选)编号计数器,默认使用自己的计数器,可以传入其他环境的name来共享计数器
% 			text:		真正在文档中打印出来的定理环境的名字
% 			section:	(可选)定理编号依赖的某个章节层次,默认不依赖。
%	利用counter共享其他环境的计数器后,autoref命令无法获取到环境名,因此引入aliascnt宏包来解决这个问题
%	具体参考https://tex.stackexchange.com/questions/187388/amsthm-with-shared-counters-messes-up-autoref-references
%
\usepackage{aliascnt}					% 定理环境编号辅助包
\newtheorem{env}{Env}[section]			% 创建一个通用的定理环境env,其他定理环境继承它的编号

\newaliascnt{theorem}{env}				% 将theorem与env的计数器相关联
\newtheorem{theorem}[theorem]{定理}	  % 新建theorem环境
\aliascntresetthe{theorem}				% 声明theorem计数器

\newaliascnt{definition}{env}
\newtheorem{definition}[definition]{定义}
\aliascntresetthe{definition}

\newaliascnt{lemma}{env}
\newtheorem{lemma}[lemma]{引理}
\aliascntresetthe{lemma}

\newaliascnt{corollary}{env}
\newtheorem{corollary}[corollary]{推论}
\aliascntresetthe{corollary}

% 例,注各自独立编号,无需考虑编号共享的问题,直接创建。证明关键词加粗
\newtheorem{example}{{例}}[chapter]
\newtheorem{remark}{{注}}[chapter]
\renewcommand{\proofname}{\bf 证明}

%=================================图表环境=====================================================


% enumitem宏包设置
\usepackage{tikz}   %矢量图工具包
\usepackage{pgfplots}
\usetikzlibrary{graphs, positioning, quotes, shapes.geometric,arrows, decorations.pathmorphing,backgrounds,fit,petri,math,calc}
\pgfplotsset{compat=1.8}
\usepackage[demo]{tikzpeople}
\usepackage{fontawesome}
\setlength{\parskip}{1ex plus 0.5ex minus 0.2ex}
\usepackage[inline]{enumitem}					% 列表工具包
\usepackage{booktabs}  %% 三线表
\usepackage{diagbox}   %% 斜线表头
\usepackage{multirow}  %% 合并单元格
\usepackage{longtable}
\usepackage{float}% 指定图片位置【H】
\usepackage{graphicx}							% 插图工具包
\usepackage{subcaption}							% 子图标题包
\usepackage{bicaption}							% 图片标题包
\setlist{%	设置列表样式
	topsep=0.2em, 			% 列表顶端的垂直空白
	partopsep=0pt, 			% 列表环境前面紧接着一个空白行时其顶端的额外垂直空白
	itemsep=0ex plus 0.1ex, % 列表项之间的额外垂直空白
	parsep=0pt, 			% 列表项内的段落之间的垂直空白
	leftmargin=1.5em, 		% 环境的左边界和列表之间的水平距离
	rightmargin=0em, 		% 环境的右边界和列表之间的水平距离
	labelsep=0.5em,         %包含标签的盒子与列表项的第一行文本之间的间隔
	labelwidth=2em, 			% 包含标签的盒子的正常宽度;若实际宽度更宽,则使用实际宽度。
}
\captionsetup[bi]{labelsep=space}
\captionsetup[figure][bi-second]{name=Figure} %设置图的英文编号前缀
\captionsetup[table][bi-second]{name=Table} %设置表的英文编号前缀
\graphicspath{figure/}		% 设置图片存放目录

%=================================代码环境=====================================================
% 使用listings宏包来插入代码
\usepackage{listings}	% 代码环境包
\renewcommand{\lstlistingname}{算法}	% 重命名代码块标题为算法,例如:算法1.2
\lstset{% 设置算法样式
	keywordstyle=\bfseries, % 设置关键词加粗
	basicstyle=\ttfamily, 	% 设置基础样式字体为等宽
	commentstyle=\ttfamily, % 基本和注释的字体都使用默认的等宽,而非texlive调用的中文字体
	showstringspaces=false, % 不显示中间的空格
	breaklines=true,  		% 对过长的代码自动换行
	frame=single ,			% 边框
}

%=================================引用工具=====================================================
% hyperref宏包教程https://www.jianshu.com/p/58e7d0a6d97a
% 实现超链接功能
\usepackage{hyperref}	% 交叉引用包
\hypersetup{%	设置交叉引用属性
	colorlinks=true,	% 设置可跳转的链接为颜色,而不是方框
	urlcolor=black,		% 设置各种链接的颜色均为黑色
	linkcolor=black,
	anchorcolor=black,
	citecolor=black
}
% 修改使用\autoref命令显示的标签前缀,renewcommand
\renewcommand*{\theoremautorefname}{定理}
\providecommand*{\definitionautorefname}{定义}
\providecommand{\lemmaautorefname}{引理}
\providecommand*{\corollaryautorefname}{推论}
\providecommand*{\exampleautorefname}{例}
\providecommand*{\remarkautorefname}{注}
\renewcommand*{\figureautorefname}{图}
\renewcommand*{\tableautorefname}{表}
\renewcommand*{\equationautorefname}{公式}
\renewcommand{\thefigure}{\arabic{chapter}-\arabic{figure}}
\renewcommand{\theequation}{\arabic{chapter}-\arabic{equation}}
\renewcommand{\thetable}{\arabic{chapter}-\arabic{table}}

\renewcommand{\eqref}[1]{\textup{{\normalfont(\ref{#1})\normalfont}}}%公式引用使用中文括号
%=================================其余设定=====================================================


%=================================正文内容=====================================================
\begin{document}
	\frontmatter                                	% 关闭章节序号与页码
		\pagestyle{empty}                           % 设置封面和原创性声明的页面样式为空
		% -*- coding=utf-8 -*-
\newcommand\clcNumber{G434}                  % 分类号
\newcommand\securityClassification{公\ 开}       % 密级
\newcommand\studentId{学号}                  % 学号
\newcommand\thesisTitle{论文题目}     % 题目
\newcommand\thesisAuthor{你的名字}                  % 作者
\newcommand\supervisor{导师名字\ 教授}               % 指导教师
\newcommand\priormajor{教育学}                   % 一级学科名称
\newcommand\minormajor{教育技术学}               % 二级学科名称
\newcommand\thesisDate{二〇二二年十二月}          % 提交日期
%============================================如非必要,以下内容请勿动==============================================
{
    \setlength\parindent{0em}                                               % 设置首行缩进为0
    \renewcommand{\baselinestretch}{1.5}\selectfont                           % 设置声明的行间距
    \ThisTileWallPaper{\paperwidth}{\paperheight}{figure/background.jpeg}    % 设置封面背景
    \vspace*{-0.6cm}                                                           % 设置垂直间距为-2,即将第一行往上移2cm
    {
        \zihao{-3}
        \begin{tabular}{l@{}c p{3cm} l@{}c}     % @{}去除边框距离,更符合学校模板的下划线紧贴描述文字
            \heiti{分\ 类\ 号}\quad     & \underline{\makebox[4cm][c]{\clcNumber}}                      \\
            \heiti{密\quad\ \  级}\quad & \underline{\makebox[4cm][c]{\kaishu \securityClassification}} \\
            \heiti{学\quad\ \  号}\quad & \underline{\makebox[4cm][c]{\studentId}}                      \\
        \end{tabular}
    }

    \vspace{7.8cm}
    {
        \centering \zihao{-3} \bfseries
        \heiti{题\quad 目}\underline{\makebox[13cm][c]{\songti \thesisTitle}}\\
        \makebox[1.7cm]{ }\underline{\makebox[13cm]{ }}
        \par
    }
    \renewcommand{\baselinestretch}{2.5}\selectfont                           % 设置声明的行间距
    \vspace{3.2cm}
    {
        \begin{center} \heiti
            {\zihao{-3} \bfseries 作\qquad \qquad 者}   \underline{\makebox[15em][c]{\kaishu\thesisAuthor}} \par
            {\zihao{-3} \bfseries 指\ \ 导\ \ 教\ \ 师} \underline{\makebox[15em][c]{\kaishu\supervisor}} \par
            {\zihao{-3} \bfseries 一级学科名称}         \underline{\makebox[15em][c]{\kaishu\priormajor}} \par
            {\zihao{-3} \bfseries 二级学科名称}         \underline{\makebox[15em][c]{\kaishu\minormajor}} \par
            {\zihao{-3} \bfseries 提\ \ 交\ \ 日\ \ 期} \underline{\makebox[15em][c]{\kaishu\thesisDate}} \par
        \end{center}
    }
}                       % 载入封面
		% -*- coding=utf-8 -*-
%============================================如非必要,以下内容请勿动==============================================
% 标题字体样式:宋体加粗二号
% 正文字体样式:宋体小四
% 正文行间距:1.45
% 标题与正文之间的垂直间距:21pt
{
\renewcommand{\baselinestretch}{2} % 设置声明的行间距
\setlength{\parindent}{2em} % 首行缩进
\vspace*{0cm}
\bfseries {
{\centering \songti \zihao{2} 学位论文原创性声明\par}    % 设置标题
\vspace{3pt}
本人声明所呈交的学位论文是我在导师的指导下进行研究工作所取得的研究成果。尽我所知,除文中已经注明引用的内容和致谢的地方外,本论文不包含其他个人或集体已经发表或撰写过的研究成果,也不包含本人或他人已申请学位或其他用途使用过的成果。对本文的研究做出重要贡献的个人和集体,均已在文中作了明确说明并表示谢意。

本学位论文若有不实或者侵犯他人权利的,本人愿意承担一切相关的法律责任。

作者签名:\underline{\makebox[7.5em][c]{}}
\quad \quad 日期:
{\makebox[3em][c]{}} 年
{\makebox[1.5em][c]{}} 月
{\makebox[1.5em][c]{}} 日\quad \quad \quad \quad \quad

}
\vspace{1.75cm}
\bfseries{
{\centering \songti \zihao{2}  学位论文知识版权及使用授权说明书\par}
\vspace{3pt}
本人在导师指导下所完成的学位论文及相关成果,知识产权归属陕西师范大学。本人完全了解陕西师范大学有关保存、使用学位论文的规定,允许本论文被查阅和借阅,学校有权保留学位论文并向国家有关部门或机构送交论文的纸质版和电子版,有权将本论文的全部或部分内容编入有关数据库进行检索,可以采用任何复制手段保存和汇编本论文。本人保证毕业离校后,发表本论文或使用本论文成果时署名单位仍为陕西师范大学。

保密论文解密后适用本声明。

作者签名:\underline{\makebox[7.5em][c]{}}
\quad \quad 日期:
{\makebox[3em][c]{}} 年
{\makebox[1.5em][c]{}} 月
{\makebox[1.5em][c]{}} 日

}
}

%===================================空白页=================================
\clearpage
\hbox{}
\clearpage                   % 载入原创性声明
		\pagenumbering{Roman}						% 切换页码至大写罗马数字显示
		% -*- coding=utf-8 -*-
\chapter{摘要}
这里是摘要!

\textbf{关键词:}{关键字1;关键字2;关键字3}
%===========================以下部分请勿动==========================
\clearpage
\thispagestyle{plain}                 % 载入中文摘要
		% -*- coding=utf-8 -*-
\chapter{Abstract}

\textbf{Keywords: }{Keywords, Keywords, Keywords.}
%===========================以下部分请勿动==========================
\clearpage
\thispagestyle{plain}                 % 载入英文摘要
		\tableofcontents                    		% 载入目录
		\clearpage									% 跳到目录下一页
		\thispagestyle{plain}						% 显示最后一页的页码
    \mainmatter                                 	% 开启章节序号计数,重置页码,页码使用阿拉伯数字
		\fancypagestyle{plain}{\pagestyle{myfancy}} % 设置默认的页面类型plain为自定义样式fancy
		\pagestyle{myfancy}							% 设置页面布局为自定义的myfancy
		% -- coding=utf-8 --
\chapter{章名}
\section{节名}
\subsection{title}
\section{节名}
\subsection{title}

\chapter{章名}
\section{节名}
\subsection{title}
\section{节名}
\subsection{title}
                    % 载入章节内容
        % -*- coding=utf-8 -*-
\chapter{总结与展望}
\section{论文工作总结}


\section{对未来工作的展望}

                  % 载入总结
    \backmatter                                 	% 关闭章节序号,对页码没有影响
		% -*- coding=utf-8 -*-
% 使用时记得删除掉\nocite
\printbibliography[title=参考文献,heading=bibintoc]
%\bibliography{bib/ref}
					% 载入参考文献
		\include{tex/10_additions}  				% 附录
	    % -*- coding=utf-8 -*-
\chapter{致谢}
听我说,谢谢你!

                      % 载入致谢
		% -*- coding=utf-8 -*-
\chapter{攻读硕士学位论文期间研究成果}
           	% 载入研究成果
\end{document}